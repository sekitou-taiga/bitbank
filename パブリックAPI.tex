\section{パブリックAPI}

\begin{table}[htbp]
  \caption{パブリックAPI}
  \label{tab:API}
  \begin{tabular}{lll}
  関数                & 説明                                                                        & 引数                                                                       \\ \hline
  get\_ticker       & 市場価格を取得                                                                   & pair                                                                     \\
  get\_depth        & 板情報を取得                                                                    & pair                                                                     \\
  get\_transactions & \begin{tabular}[c]{@{}l@{}}最新の全約定履歴を取得\\ または、\\ 指定日の全約定履歴を取得\end{tabular} & \begin{tabular}[c]{@{}l@{}}pair\\ または\\ pair, yyyymmdd=None\end{tabular} \\
  get\_candlestick  & 指定日のロウソク足データを取得                                                           & pair, candle\_type, yyyymmdd                                            
  \end{tabular}
  \end{table}


\subsection{テイカー情報の取得}
\lstinputlisting{bitbank1.py}
sell: 現在の売り注文の最安値\\
buy: 現在の買い注文の最安値\\
high: 過去24時間の最高値取引価格\\
low: 過去24時間の最安値取引価格\\
last: 最新取引価格\\
vol: 過去24時間の出来高\\

\subsection{板情報の取得}
get\_transactionsを用いることで,約定履歴が取得可能になる.
\lstinputlisting{bitbank2.py}
get\_depthを使用することで板情報が取得可能.
ここで,valueにはdict型で「asks:list型」「bids:list型」が格納されているので,joinとmapを用いて文字列に変換している.
\newpage

\subsection{全約定履歴を取得}
\lstinputlisting{bitbank3.py}
「transcation\_id」と「execued\_at」はint型なのでstr型に変換している.また,「execued\_at」はUnixTimeのミリ表示なので,日付に変更する.そこで,datetimeをインポートし,datetime.datetime.fromtimestampを用いている.日時を指定するとその日の全約定履歴も取得可能.
\begin{lstlisting}[caption=bitbank3.5,label=bitbank3.5]
  指定した日時の全約定履歴の取得
  value = get_transactions( 'ペア' )
  を変更する
  value = get_transactions( 'ペア', 'YYYYMMDD 型の日付' )
  YYYYMMDD 型の日付例⇒20180504等
\end{lstlisting}
\newpage

\subsection{ロウソク足データの取得}
get\_candlestickを用いることで,ローソク足データが取得可能となる.
\lstinputlisting{bitbank4.py}